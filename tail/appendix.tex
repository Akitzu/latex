\appendix

\chapter{Geometric Basis}

the basis vectors of the tangent space are $\{\partial_\rho, \partial_\phi, \partial_\theta\}$ and the euclidean metric become~:

$g = d\rho\otimes d\rho + (R_a+\rho\cos(\theta))^2\, d\phi\otimes d\phi + \rho^2\,d\theta\otimes d\theta $.

\begin{align*}
    \textbf{x} = (x,\,y,\,z) = (R\,\cos(\phi),\,R\,\sin(\phi),\,Z)\\
    \partial_{i'}^y = \frac{\partial}{\partial y^{i'}} = \frac{\partial x^i}{\partial y^{i'}}\frac{\partial}{\partial x^i} = \frac{\partial x^i}{\partial y^{i'}}\partial_i^x\\
    \\
    \partial_R = \cos(\phi)\,\partial_x + \sin(\phi)\,\partial_y\\
    \partial_\phi = -R\,\sin(\phi)\,\partial_x + R\,\cos(\phi)\,\partial_y\\
    \partial_Z = \partial_z
\end{align*}

\begin{align*}
    g &= dx\otimes dx + dy\otimes dy + dz\otimes dz = \delta_{ij}\,dx^i\otimes dx^j\\
    &= \delta_{ij}\frac{\partial x^i}{\partial y^{i'}}\frac{\partial x^j}{\partial y^{j'}}\,dy^{i'}\otimes dy^{j'} = \frac{\partial x^i}{\partial y^{i'}}\frac{\partial x^i}{\partial y^{j'}}\,dy^{i'}\otimes dy^{j'}\\
    &= dR\otimes dR + R^2\,d\phi\otimes d\phi + dZ\otimes dZ
\end{align*}

\begin{align*}
    \textbf{B} &= B^i\partial_i^x = B^x\partial_x + B^y\partial_y + B^z\partial_z\\
     &= B^{i'}\partial_{i'}^y =  B^R\partial_R + B^\phi\partial_\phi + B^Z\partial_Z
\end{align*}

using this formalism there is a really elegant way to describe the differential operators without the need to remind the hard formulation in each coordinate system:
\begin{align*}
    \nabla f &= (df)^\sharp\\
    \nabla\cdot F &= \star d \star (F^\flat)\\
    \nabla\times F &= (\star d (F^\flat))^\sharp\\
    \Delta f &= \star d \star d f\\
\end{align*}

\section{Integration of the action}

For the algorhithm in []. The vector potential $\textbf{A} \in T\,\mathbb{R}^3$ is $\textbf{A} =  A^i\partial_i^\text{cyl}$.

$g = dR\otimes dR + R^2\,d\phi\otimes d\phi + dZ\otimes dZ$

The curvilinear integral of the scalar product $\textbf{A}\cdot\textbf{dl}$ along $\gamma$~:
\begin{align*}
    \int_0^\phi g(\textbf{A},\textbf{dl}) = \int_0^\phi g(\textbf{A}(\gamma(s)),\dot{\gamma}(s))ds\\ = \int_0^\phi (A^R\dot{\gamma}^R + R^2A^\phi\dot{\gamma}^\phi + A^Z\dot{\gamma}^Z) ds
\end{align*}
Thus for a $q=m/n=1/0$ fixed point of an axisymmetric equilibrium $n_\text{fp} = 1$, it gives $2\pi R^2A^\phi = 2\pi R\tilde{A}^\phi$ with $\tilde{A}^\phi$ the $\phi$ component into the usual orthonormal cylindrical basis.

\chapter{Flux conservation as forms}\label{forms}

\begin{equation*}
    \iint\limits_{\Sigma}\textbf{B}\cdot\textbf{dS} = \iint\limits_{\pmap(\Sigma)}\textbf{B}\cdot\textbf{dS} \Leftrightarrow \int\limits_{\Sigma}\beta = \int\limits_{\pmap(\Sigma)}\beta = \int\limits_{\Sigma}\pmap^\star\beta
\end{equation*}

where on the on the right the equivalent differential form formalism is introduced with $\beta = B^\phi dR\wedge dZ$ the flux form and $\pmap^\star\beta$ the pullback of $\beta$ through the $F$ map.

\begin{align*}
    \beta &= \beta_{ij}dx^i\wedge dx^j = \beta_{ij}dx^i\otimes dx^j\\
    F^\star\beta &= M^{i'}_{\,\:i}M^{j'}_{\,\:j}\beta_{i'j'}dx^i\wedge dx^j = \frac{1}{2}\beta_{i'j'}\left(M^{i'}_{\,\:i}M^{j'}_{\,\:j}-M^{i'}_{\,\:j}M^{j'}_{\,\:i}\right)dx^i\otimes dx^j\\ &= \frac{1}{2}\beta_{i'j'}\delta^{i'j'}_{ij} \det(\textbf{M})dx^i\otimes dx^j = \beta_{ij}\det(\textbf{M})dx^i\otimes dx^j
\end{align*}

\chapter{Implementations}

\section{Jaxing}\label{sec:jaxpot}

\section{Jacobian of $\pmap$ in details}
The Jacobian of F as a matrix form $\dpmap = \textbf{M} = \partial F^{\{\tilde{R},\tilde{Z}\}}/{\partial \{\tilde{R},\tilde{Z}\}} \in \mathbb{R}^{2\times2}$. For instance~:
\begin{align*}
    M^{\tilde{R}}_{\,\:\tilde{R}} = \frac{\partial}{\partial \tilde{R}}\left[\int_{\phi_i}^{T+\phi_i}\dot{\gamma}^R(\phi)d\phi\right] + 1 = \int_{\phi_i}^{T+\phi_i}\partial_{\tilde{R}}\dot{\gamma}^R(\phi)d\phi + 1 = \int_0^{2\pi}\partial_{\tilde{R}}\left[\frac{B^R}{B^\phi}\right]d\phi + 1 = *
\end{align*}

\begin{align*}
  B^R = B^R(R(\phi, \tilde{R}, \tilde{Z}), \phi, Z(\phi, \tilde{R}, \tilde{Z}))\\
    B^\phi = B^\phi(R(\phi, \tilde{R}, \tilde{Z}), \phi, Z(\phi, \tilde{R}, \tilde{Z}))
\end{align*}
\begin{align*}
    \partial_{\tilde{R}}\left[\frac{B^R}{B^\phi}\right] &= \partial_{R}\left[\frac{B^R}{B^\phi}\right]\partial_{\tilde{R}}R + \partial_{Z}\left[\frac{B^R}{B^\phi}\right]\partial_{\tilde{R}}Z = \frac{1}{B^\phi}\frac{\partial B^R}{\partial\tilde{R}} - \frac{B^R}{(B^\phi)^2}\frac{\partial B^\phi}{\partial\tilde{R}} \\&= \frac{1}{B^\phi}\left(\frac{\partial B^R}{\partial R}\frac{\partial R}{\partial\tilde{R}}+\frac{\partial B^R}{\partial Z}\frac{\partial Z}{\partial\tilde{R}}\right) - \frac{B^R}{(B^\phi)^2}\left(\frac{\partial B^\phi}{\partial R}\frac{\partial R}{\partial\tilde{R}}+\frac{\partial B^\phi}{\partial Z}\frac{\partial Z}{\partial\tilde{R}}\right)
\end{align*}

In matrix form

\begin{align}
    \textbf{M} = \int_{\phi_i}^{T+\phi_i}\begin{pmatrix}
        \partial_{R}\left[B^R/B^\phi\right] & \partial_{Z}\left[B^R/B^\phi\right]\\
        \partial_{R}\left[B^Z/B^\phi\right] & \partial_{Z}\left[B^Z/B^\phi\right]
    \end{pmatrix}\cdot\begin{pmatrix}
        \partial_{\tilde{R}}R & \partial_{\tilde{Z}}R\\
        \partial_{\tilde{R}}Z & \partial_{\tilde{Z}}Z
    \end{pmatrix}d\phi + \mathbb{I}_2
\end{align}

\section{Pyoculus Newton}

Then to find a fixed point of the map we want to perform the following :
\begin{align*}
    F(\tilde{R}^*, \tilde{Z}^*) = (\tilde{R}^*, \tilde{Z}^*)\,\quad\text{with}\quad \delta\textbf{x}=(\delta \tilde{R}, \delta \tilde{Z})\\
    F(\tilde{R}^*+\delta \tilde{R}, \tilde{Z}^*+\delta \tilde{Z}) = (\tilde{R}^*, \tilde{Z}^*) + M\delta\textbf{x}+ o(\Vert \delta\textbf{x}\Vert^2)\\
    F(\tilde{R}_n,\tilde{Z}_n) = (\tilde{R}^*, \tilde{Z}^*) + M((\tilde{R}_n,\tilde{Z}_n) - (\tilde{R}^*, \tilde{Z}^*))+ o(\Vert \delta\textbf{x}\Vert^2)\\
    F(\tilde{R}_n,\tilde{Z}_n) - (\tilde{R}^*, \tilde{Z}^*)  = M(\tilde{R}_n,\tilde{Z}_n) - M(\tilde{R}^*, \tilde{Z}^*)+ o(\Vert \delta\textbf{x}\Vert^2)\\
    (\tilde{R}_{n+1},\tilde{Z}_{n+1}) = (\tilde{R}_n,\tilde{Z}_n)
\end{align*}


